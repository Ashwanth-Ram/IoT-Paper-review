\documentclass{article}
\usepackage[a4paper, portrait, margin=1.1811in]{geometry}
\usepackage[english]{babel}
\usepackage[utf8]{inputenc}
\usepackage[T1]{fontenc}
\usepackage{helvet}
\usepackage{etoolbox}
\usepackage{graphicx}
\usepackage{titlesec}
\usepackage{caption}
\usepackage{booktabs}
\usepackage{xcolor} 
\usepackage[colorlinks, citecolor=cyan]{hyperref}
\usepackage{caption}
\captionsetup[figure]{name=Figure}
\graphicspath{ {./images/} }
\usepackage{scrextend}
\usepackage{fancyhdr}
\usepackage{graphicx}
\newcounter{lemma}
\newtheorem{lemma}{Lemma}
\newcounter{theorem}
\newtheorem{theorem}{Theorem}

\fancypagestyle{plain}{
	\fancyhf{}
	\renewcommand{\headrulewidth}{0pt}
	\renewcommand{\familydefault}{\sfdefault}
	
	\lhead{\color{cyan}\small \textbf{}\\ \color{black}
	\textit{}\\ }
	%\rhead{p-ISSN: 693-7554 \\ e-ISSN:2654-3990}
	%\rfoot{\thepage} --> Show the page number
	
}

%\pagestyle{plain}
\makeatletter
\patchcmd{\@maketitle}{\LARGE \@title}{\fontsize{16}{19.2}\selectfont\@title}{}{}
\makeatother

\usepackage{authblk}
\renewcommand\Authfont{\fontsize{10}{10.8}\selectfont}
\renewcommand\Affilfont{\fontsize{10}{10.8}\selectfont}
\renewcommand*{\Authsep}{, }
\renewcommand*{\Authand}{, }
\renewcommand*{\Authands}{, }
\setlength{\affilsep}{2em}  
\newsavebox\affbox

\titlespacing\section{0pt}{12pt plus 4pt minus 2pt}{0pt plus 2pt minus 2pt}
\titlespacing\subsection{12pt}{12pt plus 4pt minus 2pt}{0pt plus 2pt minus 2pt}
\titlespacing\subsubsection{12pt}{12pt plus 4pt minus 2pt}{0pt plus 2pt minus 2pt}


\titleformat{\section}{\normalfont\fontsize{10}{15}\bfseries}{\thesection.}{1em}{}
\titleformat{\subsection}{\normalfont\fontsize{10}{15}\bfseries}{\thesubsection.}{1em}{}
\titleformat{\subsubsection}{\normalfont\fontsize{10}{15}\bfseries}{\thesubsubsection.}{1em}{}

\titleformat{\author}{\normalfont\fontsize{10}{15}\bfseries}
{\thesection}{1em}{}
\title{\textbf{\huge IoT application in weather information using wemos}\\
	}
\date{}

\begin{document}

\pagestyle{headings}	
\newpage
\setcounter{page}{1}
\renewcommand{\thepage}{\arabic{page}}


\captionsetup[figure]{labelfont={bf},labelformat={default},labelsep=period,name={Figure }}	\captionsetup[table]{labelfont={bf},labelformat={default},labelsep=period,name={Table }}
\setlength{\parskip}{0.5em}

	
\maketitle

\section{Introduction}
The Internet of Things (IOT) describes the interconnection of devices and people through the traditional internet and
social networks for various day-to-day applications like weather
monitoring, healthcare systems, smart cities, irrigation field, and
smart lifestyle. 

This paper proposes a
low-cost weather monitoring system which retrieves the weather
condition of any location from the cloud database management
system and shows the output on an OLED display. The proposed
system uses an ESP8266-EX microcontroller based Wemos D1
board and it is implemented on Arduino platform which is used
to retrieve the data from the cloud. The main objective of this
paper is to view weather conditions of any location and allows
to access the current data of any station.

\section{Ideas from the Author}
The key takeaways in this paper are:

1. Importance of weather forecasting in many fields majorly affecting human lives

2. Foundation for a smart planet

3. Utilization of cloud service for storing and retrieving data

4. Comparison of weather station services

5. Low-cost weather monitoring system prototype

\section{My Views}
In emergency times, a need for a live weather monitoring system is imminent. Hence, the goal of providing a cost-effective, simple application meets the necessities in a sudden situation.

The approach to retrieve data from weather station and displaying the information on an OLED display by using traditional internet and protocols is appealing to learn in a rudimentary way

\section{Agreements}


\section{Disagreements}



\begin{verbatim}
SUBMITTED BY:   Ashwanth Ram A S
                21011101026
                B Tech AI&DS-'A'(2nd year)
                SNU Chennai
\end{verbatim}

\end{document}